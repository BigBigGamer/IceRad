%!TEX root = ../iceDetection.tex

\documentclass[a4paper,14pt]{extarticle}

\usepackage{cmap}
\usepackage[T2A]{fontenc}
\usepackage[utf8x]{inputenc}
% \usepackage{mathptmx}
\usepackage[english, russian]{babel}

\usepackage{misccorr}
\usepackage{amssymb,amsfonts,amsmath,amsthm}  
\usepackage{indentfirst}
\usepackage[usenames,dvipsnames]{color} 
\usepackage[unicode,hidelinks]{hyperref}
\usepackage{makecell,multirow} 
\usepackage{ulem}
\usepackage{graphicx,wrapfig}
\graphicspath{{img/}}

\renewcommand{\labelenumii}{\theenumii)} 
\newcommand{\mean}[1]{\langle#1\rangle}

\DeclareMathOperator{\Div}{div}
\DeclareMathOperator{\const}{const}
%%%%%%%%%%%%%%%%%%%%%%%%%%%%%%%%%%%%%%%%%%%%%%%%%%%%%%%%%%%%%%%%%%%%%%%%%%%%%%%
%%%%%%%%%%%%%%%%%%%%%%%%%%%%%%%%%%%%%%%%%%%%%%%%%%%%%%%%%%%%%%%%%%%%%%%%%%%%%%%
\usepackage{float}
\usepackage[mode=buildnew]{standalone}
\usepackage[outline]{contour}
\usepackage{tocloft}
\renewcommand{\cftsecleader}{\cftdotfill{\cftdotsep}} % for parts
% \renewcommand{\cftchapleader}{\cftdotfill{\cftdotsep}} % for chapters
\usepackage{pgfplots,pgfplotstable,booktabs,colortbl}
\usepackage{physics}
\usepackage{mathtools}
% \mathtoolsset{showonlyrefs=true}

% \newcommand*\dotvec[1][1,1]{\crossproducttemp#1\relax}
% \def\crossproducttemp#1,#2\relax{{\qty[\vec{#1}\times\vec{#2}\,]}}

% \newcommand*\prodvec[1][1,1]{\crossproducttempa#1\relax}
% \def\crossproducttempa#1,#2\relax{{\qty[{#1}\times{#2}\,]}}
% \usepackage{showframe}
\usepackage[]{geometry}
\geometry{
  left=2.5cm,
  right=1.5cm,
  top=2cm,
  bottom=2cm,
  bindingoffset=0cm,
  headheight=17pt
}
\linespread{1.5} 
\setlength{\parindent}{1.25cm}
\frenchspacing 
\usepackage{setspace}
\setlength{\tabcolsep}{20pt}
\renewcommand{\arraystretch}{1.5}
\usepackage{xcolor}



\begin{document}

\section*{Введение}
\subsection*{Актуальность}
Про лед, как он исследуется, как он важен, и тд.

\subsection*{Методы исследования}
Спутники, спутнки, а на них радары, радары.

\subsection*{Цель работы}
Разработка методов распознования льда по данным для малых углов падения. 

\section*{Миссия GPM}
Что это, зачем это, для чего это. Подробное описание аппаратуры и радара.

Данные для малых углов падения.

В нашей работе использовались данные DPR, Ku-диапазона.

\section*{Теоретическое введение}

Что измеряет радар, что такое УЭПР, в каком приближении мы работаем.

Какое УЭПР имеет вода, немного выводов формул.

\section*{Разработка метода детектирования}

\subsection*{Первоначальный анализ данных}
Рассмотрим пример трека, проходящего через ледяной покров. Рассмотрим поперечные зависимости, сделаем вывод, что лед
отличается от воды угловой зависимостью.

\subsection*{Коэффициент эксцесса}
Что такое, как считается, почему используется(для выявления пика распределения)

Описание алгоритма расчета 

Первоначальное приближение

\subsection*{Границы}
Формулы, как считаются, зачем, примеры посчитанных границ трека.

Новые карты льда с использованием границ.

\section*{Выводы}
\section*{Литература}


\end{document}